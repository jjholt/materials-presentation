\usepackage[T1]{fontenc}
\usepackage[utf8]{inputenc}

%Maths
\usepackage{pgfplots}
\usepackage{pgfplotstable, booktabs, array}
	\pgfplotsset{compat=newest}
	\usepgfplotslibrary{patchplots}
	\pgfplotsset{plot coordinates/math parser=false}
\usepackage{amssymb, amsmath}
\usepackage{cancel}
\usepackage{tikz}
	\usetikzlibrary{plotmarks}
	\usetikzlibrary{arrows.meta}
	\usetikzlibrary{positioning,shapes,arrows}
	%Control blocks
	\tikzstyle{block} = [draw, minimum width = 2cm, minimum height = 1.2cm, fill=blue!5]
	\tikzstyle{sum} = [draw, fill=blue!5, circle, node distance=1cm]
\newcommand{\texdim}[2]{%Import and scale matlab2tikz images
	\newlength\fheight
	\newlength\fwidth
    \setlength\fheight{#1}
    \setlength\fwidth{#2}
}
\usepackage{siunitx} %SI units formatting

%Electric circuits
\usepackage{circuitikz}

%Organisation
\usepackage{import}
\usepackage[backend=biber, style=verbose]{biblatex}
% \usepackage[square, numbers]{natbib} %Bibliography style
% \bibliographystyle{unsrtnat} %Bibliography style
\usepackage{pdfpages}
\usepackage{transparent}
\usepackage{xcolor}

%Importing inkscape images with \incfig
\newcommand{\incfig}[2][1]{%
    \def\svgwidth{#1\columnwidth}
    \import{./figures/}{#2.pdf_tex}
}

%Code
\usepackage{listings}
\lstset{
	numbers=left, frame=single, breaklines=true, %Keep text inside a frame, and number each line.
	basicstyle = \scriptsize\ttfamily, %smaller size, monospaced
}

% Miscellanious
\definecolor{dark-red}{RGB}{96, 25, 46}


\setbeamertemplate{headline}{%
	\includegraphics[width=\paperwidth]{figures/Dark-Red}
}
\setbeamercolor{footline}{bg=dark-red}
\setbeamercolor{frametitle}{fg=white}
\setbeamertemplate{frametitle}{
	\vspace{-0.6cm}
	\insertframetitle
}
\usepackage{xcolor}
\beamertemplatenavigationsymbolsempty